% Skeleton report for Assignment X1, in Reiknirit, fall 2017, at Reykjavik University
%  (c) 2017, Magnus M. Halldorsson
%
% Builds on the assignment Percolation at Princeton University, (c) Kevin Wayne
%
% Format is based on the following document:
% A basic LaTeX document for a handin with a standard RU title page
%  (c) 2013, Tómas Ken Magnússon, 

% If you want the title to appear on a separate page, change notitlepage to titlepage
\documentclass[11pt,a4paper,notitlepage]{article}
\usepackage[utf8]{inputenc}
\usepackage[T1]{fontenc}
% If your hand-in is in icelandic change english to icelandic
% Note: This has nothing to do with Icelandic characters, they
% can always be used. This just tells other packages what
% language you are using and changes the hyphenation used by LaTeX
% If icelandic is selected2 a shorthand, "` and "', is also included
% for Icelandic quotation marks. They can also obtained by using
% ,, and ``
\usepackage[english]{babel}
\usepackage{amsmath, amsthm, amssymb, amsfonts}
\usepackage{graphicx}
\usepackage{enumerate}
% To use the whole A4-page
% See: ftp://ftp.tex.ac.uk/tex-archive/macros/latex/contrib/geometry/geometry.pdf
% and http://en.wikibooks.org/wiki/LaTeX/Document_Structure
\usepackage{geometry}
% For header and footer
% See: ftp://ctan.tug.org/tex-archive/macros/latex/contrib/ fancyhdr/fancyhdr.pdf
% and http://en.wikibooks.org/wiki/LaTeX/Document_Structure
\usepackage{fancyhdr}
% For prettier tables
% See: http://ctan.mackichan.com/macros/latex/contrib/booktabs/booktabs.pdf
% and  http://en.wikibooks.org/wiki/LaTeX/Tables
%\usepackage{booktabs}
\usepackage{listings} % For code listing

% \usepackage{sagetex}
\usepackage{hyperref}
\usepackage{caption}
\usepackage[usenames,dvipsnames,svgnames,table]{xcolor}
\usepackage{subfig}  % To fit two tables side-by-side

%%%%%%%%%%%%%%%%%%%%%%%%%%%%%%%%%%%%%%%%%%%%%%%%%%%%%%%%%%%%%
%                        Setup
%%%%%%%%%%%%%%%%%%%%%%%%%%%%%%%%%%%%%%%%%%%%%%%%%%%%%%%%%%%%%

% Set the margins of the paper. By default LaTeX uses huge margins
\geometry{includeheadfoot, margin=2.5cm}
% you can also use
% \geometry{a4paper}
% End of margins setup

% Settings for listings
\lstset{language=Java,numbers=left,backgroundcolor=\color{light-gray},
        basicstyle=\scriptsize\ttfamily,frame=single,tabsize=4,
        captionpos=t, numbers=left,
        keywordstyle=\color{javapurple}\bfseries,
        stringstyle=\color{javared},
        commentstyle=\color{javagreen},
        morecomment=[s][\color{javadocblue}]{/**}{*/},}
% End of settings for listings

% Custom colors for listings
\definecolor{light-gray}{gray}{0.95}
\definecolor{javared}{rgb}{0.6,0,0} % for strings
\definecolor{javagreen}{rgb}{0.25,0.5,0.35} % comments
\definecolor{javapurple}{rgb}{0.5,0,0.35} % keywords
\definecolor{javadocblue}{rgb}{0.25,0.35,0.75} % javadoc
% End of custom colors for listings


% Fill in any relevant information
% Leave the fields inside the {} empty if they do not apply
\newcommand{\semester}{Fall 2017}
\newcommand{\coursename}{Reiknirit}
\newcommand{\courseid}{T-301-REIR}
\newcommand{\assignment}{X1: Percolation}
\newcommand{\problemtitle}{Problem}
\newcommand{\dateofcompilation}{\today}

%% Information about you --- FILL THIS OUT
\newcommand{\ssn}{kt. 123456-7890}              %%%  CHANGE THIS """
\newcommand{\group}{1}
\newcommand{\teachingassistant}{TA: Eiríkur Fjalar}   %%%  CHANGE THIS """
\newcommand{\students}{
    Name of Student                             %%%  CHANGE THIS """
}
\newcommand{\studentemail}{
    myEmails@ru.is                            %%%  CHANGE THIS """
}

% Setup header and footer
% Headers
\pagestyle{fancy} % To get the header and footer
\chead{\small \textsc{\assignment}}
\rhead{\small \textsc{\coursename}}
\lhead{\small \textsc{\studentemail}}
% Footers
%\lfoot{Left footer text}
%\cfoot{\thepage} % This is the default behaviour
%\rfoot{Right footer text}


% Custom dot for itemize
\renewcommand{\labelitemi}{$\cdot$}

% If you don't want a line below the header or above the footer,
% change the appropriate header/footerrulewidth to 0pt
\setlength{\headheight}{15.2pt} % This is set to avoid a warning
\renewcommand{\headrulewidth}{0.4pt}
\renewcommand{\footrulewidth}{0.4pt}
% End of header and footer setup

% Setup Problem/Solution environments // You probably don't need this
%\theoremstyle{plain}
%\newtheorem{problem}{Dæmi}
%\theoremstyle{remark}
%\newtheorem*{solution}{Lausn}
% End of Problem/Solution environments setup
%\theoremstyle{plain}
%\newtheorem*{proposition}{Proposition}

\DeclareCaptionLabelFormat{andtable}{#1~#2  \&  \tablename~\thetable}

% Custom problem (so you can provide the problem name)
%\newenvironment{cproblem}[1]{\begin{trivlist}
%\item[\hskip \labelsep {\bfseries \problemtitle}\hskip \labelsep {\bfseries#1.}]\begin{itshape}}{\end{itshape}\end{trivlist}}
% End of Problem/Solution environments setup

% The title page
\newcommand{\maketitlepage}[1]
{
    \begin{titlepage}

        \begin{center}
            \includegraphics[width=0.55\textwidth]{./rulogo.png}\\[1.5cm]

            \textsc{\huge \semester}\\[0.8cm]

            {\textsc{\Huge \courseid, \coursename}}\\[0.4cm]
            \textsc{\LARGE }\\[2.5cm]

            \textbf{\textsc{\Huge #1}}\\[3cm]


            \textsc{\huge \students}\\[0.4cm]
            \textsc{\LARGE \ssn}\\[0.4cm]
            \textsc{\LARGE Group \group}\\[1cm]
            \textsc{\Large \dateofcompilation}


        \end{center}

        \vfill

        % You may also want to add the name of your teaching assistant
        \begin{flushleft}
            \textsc{\Large \teachingassistant}
        \end{flushleft}

    \end{titlepage}
}
\newcommand{\command}[1]{\texttt{\textbackslash{}#1}}

\newcommand{\explanation}[1]{}  %% Use this when turning in the report
%\newcommand{\explanation}[1]{
\begin{quote}\emph{#1} \end{quote}}  %% Use this to include directions

%%%%%%%%%%%%%%%%%%%%%%%% END OF SETUP %%%%%%%%%%%%%%%%%%%%%%%%


\begin{document}
% Create the title page
    \maketitlepage{\assignment}

\explanation{Directions on performing the assignment are showed here in italics (like this). These should not be included in the report you submit.}

 \section{Introduction}
\explanation{  State the objective(s) of the exercise. Ask yourself: Why did I perform the
  experiment? What did I aim to achieve? Provide background about the subject
  matter, as needed (what are union-find data structures good for?). Include
  the purpose of the different equipment and steps.
}

  \subsection{Setup and Methods}
\explanation{ Describe how you performed the exercise. Write about what you actually did
  rather than what you were supposed to do. Be concise. Only give the
  necessary details a person in the same field needs to perform the exercise.
  Write in narrative form (i.e., telling a story) rather than a numbered list
  format. Describe both the set-up (hardware, OS, software, tools) and the testing
  process. Refer to the classes written, but do not include them in the report.}



\subsection{Implementation}
% /******************************************************************************
\explanation{Describe how you implemented Percolation.java. How did you check
whether the system percolates?}
%  *****************************************************************************/


\section{Results}
\explanation{Describe the experimental results briefly in words, referring to the tables.}

\subsection{Results With Quick-Find}

% /******************************************************************************
\explanation{Using Percolation with \texttt{QuickFindUF.java}, fill in the table below such that the $N$ values are
  multiples of each other.  Also fill in the second table, using a fixed, relevant value of $N$}
%  *****************************************************************************/

\begin{table}[htbp]
%  \small
  \centering
  \caption{Caption for both}
  \subfloat[Insert caption ($T$ fixed)]{   %% CHANGE THE CAPTIONS
        \label{tab:table1}
        \begin{tabular}{|r| c |}
        \hline
        $N$ & Running time \\ 
        \hline
        $\ldots$ & $\ldots$ \\
        $\ldots$ & $\ldots$ \\
        $\ldots$ & $\ldots$ \\
        $\ldots$ & $\ldots$ \\
        $\ldots$ & $\ldots$ \\
        $\ldots$ & $\ldots$ \\
        \hline
        \end{tabular}
  }
\qquad \qquad
  \subfloat[Insert caption ($N$ fixed)]{
        \label{tab:table1}
        \begin{tabular}{|r| c |}
        \hline
        $T$ & Running time \\    %% Fixed, 31 August 2016 / MMH
        \hline
        $\ldots$ & $\ldots$ \\
        $\ldots$ & $\ldots$ \\
        $\ldots$ & $\ldots$ \\
        $\ldots$ & $\ldots$ \\
        $\ldots$ & $\ldots$ \\
        $\ldots$ & $\ldots$ \\
        \hline
        \end{tabular}
  }
\end{table}


\explanation{Refer to these table in the text!}

\paragraph{Time complexity}
\explanation{
Give a formula (using tilde notation) for the running time (in seconds) of 
\texttt{PercolationStats.java} as a function of both $N$ and $T$. Be sure to give both 
the coefficient and exponent of the leading term. Your coefficients should 
be based on empirical data and rounded to two significant digits, such as 
$5.3*10^{-8} \cdot N^5.0 T^{1.5}$.}

Running time as a function of $N$ and $T$:  $\sim$ 

\explanation{Explain briefly how you come up this running time}
% \textbf{Reasoning:}

\subsection{Results with Weighted Quick-Union}
% /******************************************************************************
\explanation{Repeat the previous question, but use WeightedQuickUnionUF.java.}
%  *****************************************************************************/

\begin{table}[htbp]
%  \small
  \centering
  \caption{Caption for both}
  \subfloat[Insert caption ($T$ fixed)]{
        \label{tab:table1}
        \begin{tabular}{|r| c |}
        \hline
        $N$ & Running time \\ 
        \hline
        $\ldots$ & $\ldots$ \\
        $\ldots$ & $\ldots$ \\
        $\ldots$ & $\ldots$ \\
        $\ldots$ & $\ldots$ \\
        $\ldots$ & $\ldots$ \\
        $\ldots$ & $\ldots$ \\
        \hline
        \end{tabular}
  }
\qquad \qquad
  \subfloat[Insert caption ($N$ fixed)]{
        \label{tab:table1}
        \begin{tabular}{|r| c |}
        \hline
        $N$ & Running time \\ 
        \hline
        $\ldots$ & $\ldots$ \\
        $\ldots$ & $\ldots$ \\
        $\ldots$ & $\ldots$ \\
        $\ldots$ & $\ldots$ \\
        $\ldots$ & $\ldots$ \\
        $\ldots$ & $\ldots$ \\
        \hline
        \end{tabular}
  }
\end{table}


% * Refer to these table in the text!

\paragraph{Time complexity}
\explanation{
Give a formula (using tilde notation) for the running time (in seconds) of 
\texttt{PercolationStats.java} as a function of both $N$ and $T$. Be sure to give both 
the coefficient and exponent of the leading term. Your coefficients should 
be based on empirical data and rounded to two significant digits, such as 
$5.3*10^{-8} \cdot N^5.0 T^{1.5}$.}

Running time as a function of $N$ and $T$:  $\sim$ 

\explanation{Explain briefly how you come up this running time}
% \textbf{Reasoning:}


\subsection{Memory Usage}
% /**********************************************************************
\explanation{
How much memory (in bytes) does a \texttt{Percolation} object use to store
an $N$-by-$N$ grid? Use the 64-bit memory cost model from Section 1.4
of the textbook and use tilde notation to simplify your answer.
Briefly justify your answers.}

\explanation{Include the memory for all referenced objects (deep memory).}
%  **********************************************************************/




\section{About This Solution}

Have you taken (part of) this course before:

Hours to complete assignment (optional):


% /******************************************************************************
%  *  After reading the course rules on collaboration policy, answer the
%  *  following short quiz. This counts for a portion of your grade.
%  *  Write down the answers in the space below.
%  *****************************************************************************/
\subsection{Quiz on Collaboration}

\begin{enumerate}
\item How much help can you give a fellow student taking REIR?
\begin{enumerate}
\item None. Only the TAs can help.
\item You can discuss ideas and concepts but students can get help
    debugging their code only from a TA, or
    student who has already passed REIR.
\item You can help a student by discussing ideas, selecting data
    structures, and debugging their code.
\item You can help a student by emailing him/her your code.
\end{enumerate}

\textbf{Answer:} 

%\begin{enumerate}
\item What is the expectation when partnering?
\begin{enumerate}
 \item You and your partner split the assignment between you and solve it individually.
 \item You and your partner discuss all the problems together, but code individually.
 \item You and your partner discuss the problems and write all the code together.
\end{enumerate}
\end{enumerate}

\textbf{Answer:} 
 
\subsection{Known Bugs / Limitations.}
% /******************************************************************************
%  *  Known bugs / limitations.
%  *****************************************************************************/

\subsection{Help Received}
% /******************************************************************************
\explanation{
Describe whatever help (if any) that you received.
Don't include readings, lectures, and classes, but do
include any help from people (including course staff, lab TAs,
classmates, and friends) and attribute them by name.}
%  *****************************************************************************/


\subsection{Problem Encountered}
% /******************************************************************************
\explanation{
Describe any serious problems you encountered.                    }
%  *****************************************************************************/



\subsection{Comments}
% /******************************************************************************
\explanation{
List any other comments here. Feel free to provide any feedback   
on how much you learned from doing the assignment, and whether    
you enjoyed doing it.}
% *****************************************************************************/




\end{document}

%%%% Optional material

% Example plot
%    \begin{figure}[!ht]
%        \label{fig:plot1}
%        \centering
%        \includegraphics[width=0.6\textwidth]{chart.pdf}
%        \caption{This is an imported figure}
%    \end{figure}

% Example reference:    See Listing \hyperref[lst:random]{1}.

    \pagebreak
    \section*{Appendix I: Source code listings}
    Optional.

    \subsection*{Acknowledgement}

    % Environment for listing code
    \begin{lstlisting}[caption={This is a caption.},label={lst:array1}]
public class This is a class {

    // This is a comment
    public static double thisIsAFunction(int N) {
        QuickUnionUF uf = new QuickUnionUF(N);
        return something;
    }
    public static void main(String[] args) { 
        int T = 50;
    }
} 
    \end{lstlisting}


    \pagebreak
    \section*{Appendix II}
    Optional.

    \listoftables
    \listoffigures
    \lstlistoflistings
    \bibliographystyle{plain}
    \bibliography{s1.bib}

\end{document}

